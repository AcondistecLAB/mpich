% 
%   This is a latex file that generates a reference manual for the 
%   Model MPI implementation
%
\documentclass{article}
\usepackage[dvipdfm]{hyperref}
\usepackage{refman}
\usepackage{tpage}
\textheight=9in
\textwidth=6.1in
\oddsidemargin=.2in
\topmargin=-.50in

% Include each routine in the contents page
\mancontentstrue

\begin{document}

\markright{MPICH2 Reference Manual}


\def\nopound{\catcode`\#=13}
{\nopound\gdef#{{\tt \char`\#}}}
%\catcode`\_=13
%\def_{{\tt \char`\_}}
\catcode`\_=11
\def\code#1{{\tt #1}}

\ANLTitle{MPICH2 Model MPI Implementation\\Reference Manual\\\ \\Draft}{\em 
William Gropp\\
Ewing Lusk
Mathematics and Computer Science Division\\
Argonne National Laboratory}{00}{\today}

\clearpage

\pagenumbering{roman}
\tableofcontents
\clearpage

\pagenumbering{arabic}
\pagestyle{headings}

\section{Introduction}
This document contains detailed documentation on the routines that are part of
the MPICH model MPI implementation.
As an alternative to this manual, the reader should consider using either the
man pages (often installed in \file{/usr/local/mpi/man}) or the web pages
(often installed in \file{/usr/local/mpi/www}).

%As an alternate to this manual, the reader should consider using the
%script \code{mpiman}; this is a script that uses \code{xman} to provide
%a X11 Window System interface to the data in this manual.

\section{MPI Commands}
\input refcmd.tex

\section{MPI routines}
This section contains descriptions of each of the routines in the MPI
standard, including both MPI-1 and MPI-2.

In the description of the routine parameters, the following terms are
used:
\begin{description}
\item[handle]An MPI object handle.  This is simply an object of the
  specified type.
\item[choice]Any type.  For example, this may be an 'int', 'double',
  or structure type.  These are always passed with a pointer and in
  the C and C++ bindings, a \code{void*} pointer is used.
\end{description}
Other terms, such as ``nonnegative integer,'' are self explanatory.

\input refmpi.tex

%\section{MPE routines}
%\input ref4/ref.tex

%\section{ADI routines}
%\input ref5/ref.tex

\end{document}
